\documentclass[A4,svgnames,9pt,aspectratio=169]{beamer}
%% document options:
%% - aspectratio = { 43, 169, 1610 }
%% - utf8
%%

%%
%% insert list of packages
%%

%% \usepackage[english,french]{babel}

\hypersetup{ 
   allcolors=bleu_utt_clair,
   pdfauthor   = {firstname lastname},
   pdftitle    = {\@title},
   pdfsubject  = {Resum\'{e} of everything},
   pdfkeywords = {firstname~lastname, curriculum vit\ae{}}
}

%%%%%%%%%%%%%%%%%%%%%%%%%%%%%%%%%%%%%%%%%%%%%%%%%%%%%%%
%%
%%%%%%%%%%%%%%%%%%%%%%%%%%%%%%%%%%%%%%%%%%%%%%%%%%%%%%%

\title[titrecourt]{Soutenance de Stage}
\subtitle{Sujet du stage\\ sous-sujet pour la soutenance}
\date[00/00/202X]{date long}
\author[A. et al.]{Auteur(s)}
\newcommand{\semester}{A24}
\newcommand{\course}{UE name}

\usetheme{utt}

\begin{document}

%%%%%%%%%%%%%%%%%%%%%%%%%%%%%%%%%%%%%%%%%%%%%%%%%%%%%%%
%%
%%%%%%%%%%%%%%%%%%%%%%%%%%%%%%%%%%%%%%%%%%%%%%%%%%%%%%%

\frame{\titlepage}

%%%%%%%%%%%%%%%%%%%%%%%%%%%%%%%%%%%%%%%%%%%%%%%%%%%%%%%

% Le titre des planches de sommaire est \contentsname, sa valeur
% est fixée ici à "Sommaire" par défaut.
\renewcommand{\contentsname}{Sommaire}

% Si le package babel est utilisé la valeur
% prend celle par défaut de la langue sélectionnée.
% Pour imposer une valeur, y compris lors de l'utilisation de
% babel, faire attention à redéfinir cette variable _après_
% un \selectlanguage{}

% \selectlanguage{french}
% \selectlanguage{english}

\frame{\tocpage}

%%%%%%%%%%%%%%%%%%%%%%%%%%%%%%%%%%%%%%%%%%%%%%%%%%%%%%%
 
\section{Titre du chapitre 1}
\frame{\sectionpage}

%%%%%%%%%%%%%%%%%%%%%%%%%%%%%%%%%%%%%%%%%%%%%%%%%%%%%%%

\begin{frame}{Titre de la slide}
    \begin{block}{Premier niveau de liste (Titre)}
       Deuxième niveau de liste (Texte courant). Ut wisi enim ad minim veniam, quis nostrud exerci tation
       ullamcorper suscipit loborti. Mauris tempor adipiscing ligula bibendum. Vestibulum sapien lectus,
       porttitor vel euismod a, lobortis at mauris.
      \begin{itemize}
         \item Troisième niveau de liste (Puce 1). Vestibulum sapien lectus, porttitor vel euismod a.
         \item Troisième niveau de liste (Puce 1). Vestibulum sapien lectus, porttitor vel euismod a.
         \begin{itemize}
            \item Quatrième niveau de liste (Puce 2). Netus et malesuada fames ac turpis egestas.
            \item Quatrième niveau de liste (Puce 2). Netus et malesuada fames ac turpis egestas.
            \begin{itemize}
               \item Cinquième niveau de liste (Puce 2). Mauris tempor turpis eu libero sollicitudin.
               \item Cinquième niveau de liste (Puce 2). Mauris tempor turpis eu libero sollicitudin.
            \end{itemize}
          \end{itemize}
        \end{itemize}
   \end{block}
\end{frame}

\subsection{Section x.x}

\begin{frame}{Titre de la slide \newline sur 2 lignes}
    \begin{block}{Premier niveau de liste (Titre)}
       Deuxième niveau de liste (Texte courant). Ut wisi enim ad minim veniam, quis nostrud exerci tation
       ullamcorper suscipit loborti. Mauris tempor adipiscing ligula bibendum. Vestibulum sapien lectus,
       porttitor vel euismod a, lobortis at mauris.
      \begin{enumerate}
         \item Premier niveau \texttt{enumerate} (Puce 1). 
         \item Premier niveau \texttt{enumerate} (Puce 1). 
         \begin{enumerate}
            \item Deuxième niveau \texttt{enumerate} (Puce 2).
            \item Deuxième niveau \texttt{enumerate} (Puce 2).
            \begin{enumerate}
               \item Troisième niveau \texttt{enumerate} (Puce 3).
               \item Troisième niveau \texttt{enumerate} (Puce 3).
            \end{enumerate}
          \end{enumerate}
        \end{enumerate}
   \end{block}
\end{frame}

%%%%%%%%%%%%%%%%%%%%%%%%%%%%%%%%%%%%%%%%%%%%%%%%%%%%%%%

\section{Titre du chapitre 2}
\frame{\sectionpage}

%%%%%%%%%%%%%%%%%%%%%%%%%%%%%%%%%%%%%%%%%%%%%%%%%%%%%%%

\begin{frame}{Mise en page sur 2 colonnes}
  \begin{columns}
      
      \begin{column}[t]{0.45\textwidth}
          \begin{center}
          \includegraphics[width=0.95\textwidth]{imgs/photo.png}      
          \end{center}
      \end{column}
        
    \begin{column}[t]{0.45\textwidth}

           \begin{block}{Premier niveau de liste (Titre)}
           Deuxième niveau de liste (Texte courant).
             \begin{itemize}
               \item Troisième niveau de liste (Puce 1).
                   \begin{itemize}
                     \item Quatrième niveau de liste (Puce 2).
                        \begin{itemize}
                           \item Cinquième niveau de liste (Puce 3).
                        \end{itemize}
                   \end{itemize}
             \end{itemize}
           \end{block}

           \begin{block}{Premier niveau de liste (Titre)}
           Deuxième niveau de liste (Texte courant).
             \begin{itemize}
               \item Troisième niveau de liste (Puce 1).
                   \begin{itemize}
                     \item Quatrième niveau de liste (Puce 2).
                   \end{itemize}
             \end{itemize}
           \end{block}

         \end{column}
         
  \end{columns}
\end{frame}

%%%%%%%%%%%%%%%%%%%%%%%%%%%%%%%%%%%%%%%%%%%%%%%%%%%%%%%

\section{Titre du chapitre 3}
\frame{\sectionpage}

%%%%%%%%%%%%%%%%%%%%%%%%%%%%%%%%%%%%%%%%%%%%%%%%%%%%%%%

\begin{frame}{Deux autres exemples de blocs}
    \begin{exampleblock}{}
      \begin{itemize}
        \item{Item 1 }
        \item {Item 2}
        \item {Item 3}
      \end{itemize}
    \end{exampleblock}
    \begin{exampleblock}{Avec titre}
      \begin{itemize}
        \item{Item 1 }
        \item {Item 2}
        \item {Item 3}
      \end{itemize}
    \end{exampleblock}
\end{frame}

%%%%%%%%%%%%%%%%%%%%%%%%%%%%%%%%%%%%%%%%%%%%%%%%%%%%%%% 

\section{Conclusion}
\begin{frame}{Conclusion}

    {\tiny         The quick brown fox jumps over the lazy dog}\\
    {\scriptsize   The quick brown fox jumps over the lazy dog}\\
    {\footnotesize The quick brown fox jumps over the lazy dog}\\
    {\small        The quick brown fox jumps over the lazy dog}\\
    {\normalsize   The quick brown fox jumps over the lazy dog}\\
    {\large        The quick brown fox jumps over the lazy dog}\\
    {\Large        The quick brown fox jumps over the lazy dog}\\
    {\LARGE        The quick brown fox jumps over the lazy dog}\\
    {\huge         The quick brown fox jumps over the lazy dog}\\
    {\Huge         The quick brown fox jumps over the lazy dog}

\end{frame}

%%%%%%%%%%%%%%%%%%%%%%%%%%%%%%%%%%%%%%%%%%%%%%%%%%%%%%%

%% Le texte est modifiable en changeant \thankyou
%% \renewcommand{\thankyou}{Thank You.}
\frame{\merci}

%%%%%%%%%%%%%%%%%%%%%%%%%%%%%%%%%%%%%%%%%%%%%%%%%%%%%%% 

\end{document}

%%%%%%%%%%%%%%%%%%%%%%%%%%%%%%%%%%%%%%%%%%%%%%%%%%%%%%%
%%
%%%%%%%%%%%%%%%%%%%%%%%%%%%%%%%%%%%%%%%%%%%%%%%%%%%%%%%

